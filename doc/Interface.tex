\documentclass{scrartcl}

\usepackage[T1]{fontenc}
\usepackage[utf8]{inputenc}
\usepackage[english]{babel}
\usepackage[table]{xcolor}
\usepackage[labelfont=bf]{caption}
\usepackage[colorinlistoftodos,%
			prependcaption,%
			textsize=tiny]{todonotes}
\usepackage{hyperref}
\usepackage{enumitem}
\usepackage{multicol}
\usepackage{multirow}
\usepackage{tabularx}
\usepackage{nameref}
\usepackage{floatrow}
\usepackage[outputdir=build]{minted}
\usepackage{url}
\usepackage{tikz}
\usepackage{csquotes}
\usepackage{pdflscape}
\usepackage{environ}
\usepackage{booktabs}
\usepackage[citestyle=alphabetic-verb,%
			bibstyle=numeric,%
			hyperref,%
			backend=biber]{biblatex}
\usepackage{pgfgantt}

%% Meta Information %%%%%%%%%%%%%%%%%%%%%%%%%%%%%%%%%%%%%%%%%%%%%%%

\newcommand{\thetitle}{Algorithm Engineering - Parallel Graph Colouring Framework}

\newcommand{\thekeywords}{Algorithm Engineering SS 17}

\renewcommand{\theauthor}{Rudolf Biczok}

\newcounter{myWeekNum}
\stepcounter{myWeekNum}
%
\newcommand{\myWeek}{\themyWeekNum
	\stepcounter{myWeekNum}
	\ifnum\themyWeekNum=53
	\setcounter{myWeekNum}{1}
	\else\fi
}

\setlength\parindent{0pt}

%% PDF Setup %%%%%%%%%%%%%%%%%%%%%%%%%%%%%%%%%%%%%%%%%%%%%%%%%%%%%%

\hypersetup{
	% non-Latin characters in Acrobat’s bookmarks
	unicode=true, %
	% show Acrobat’s toolbar?
	pdftoolbar=true, %
	% show Acrobat’s menu?
	pdfmenubar=true, %
	% window fit to page when opened
	pdffitwindow=false, %
	% fits the width of the page to the window
	pdfstartview={FitH}, %
	% title
	pdftitle={\thetitle}, %
	% author
	pdfauthor={\theauthor}, %
	% subject of the document
	pdfsubject={Industry 4.0}, %
	% creator of the document
	pdfcreator={\theauthor}, %
	% producer of the document     
	pdfproducer={Producer}, %
	% keywords
	pdfkeywords={\thekeywords}, %
	% links in new window
	pdfnewwindow=false, %
	% false: boxed links; true: colored links
	colorlinks=true, %
	% color of internal links (change box color with linkbordercolor)  
	linkcolor=blue, %
	% color of links to bibliography
	citecolor=green, %
	% color of file links
	filecolor=magenta, %
	% color of external links
	urlcolor=cyan %
}


%% Actual Content %%%%%%%%%%%%%%%%%%%%%%%%%%%%%%%%%%%%%%%%%%%%%%%%%

\begin{document}

\section{Operators}

The operators you are going to implement should have the following signature:

\subsection{Initialization Operator}

\renewcommand{\theFancyVerbLine}{\sffamily \textcolor[rgb]{0.5,0.5,1.0}{\normalsize \oldstylenums{\arabic{FancyVerbLine}}}}

\begin{minted}[frame=lines,
               framesep=2mm,
               baselinestretch=1.2,
               linenos]{cpp}
Colouring initOperator(const graph_access &G, const ColorCount k);
\end{minted}

\textbf{Returns} the vector of unsigned int values \\ (\mintinline{cpp}{typedef std::vector<uint32_t> Colouring}) representing the colouring for each node.

Use the marko \mintinline{cpp}{UNCOLORED} to mark a node as uncolored.

\textbf{Parameters:}
\begin{description}
	\item[graph\_access] The input graph used to generate the colourings
	\item[k] The number of colours passed from parallel algorithm. \\
	It can be ignored if the algorithm does not need it for initialization.
\end{description}

\subsection{Crossover Operator}

\begin{minted}[frame=lines,
               framesep=2mm,
               baselinestretch=1.2,
               linenos]{cpp}
Colouring crossoverOperator(const Colouring &s1, const Colouring &s2
                            const graph_access &G);
\end{minted}

\textbf{Returns} a new colouring based on two given parent colourings.

\textbf{Parameters:}
\begin{description}
	\item[s1 and s2] The two parent colourings
	\item[G] The input graph used to generate the colourings
\end{description}

\subsection{Local Search Operator}

\begin{minted}[frame=lines,
               framesep=2mm,
               baselinestretch=1.2,
               linenos]{cpp}
Colouring lsOperator(const Colouring &s, const graph_access &G);
\end{minted}

\textbf{Returns} a enhanced variant of the colouring s.

\textbf{Parameters:}
\begin{description}
	\item[s] The source colourig to mutate
	\item[G] The input graph used to generate the colourings
\end{description}

\end{document}
